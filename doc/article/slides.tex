\documentclass[10pt, compress, xcolor={usenames,dvipsnames}]{beamer}
\usepackage[utf8x]{inputenc}

%%% Theme and style
\usetheme{metropolis}
\metroset{progressbar=frametitle}
\metroset{numbering=fraction}

%%% Packages %%%

% Use T1 and a modern font family for better support of accents, etc.
\usepackage[T1]{fontenc}
\usepackage{palatino}  % Palatino

% Language support
\usepackage[english]{babel}

% Support for easily changing the enumerator in
% enumerate-environments.
\usepackage{enumerate}

% Support for importing images
%\usepackage{graphicx}

% Use hyperlinks
\usepackage{hyperref}

% Don't load xcolors package in beamer: use document class option
% instead...
%\usepackage[usenames,dvipsnames]{xcolor}

% Use colors in tables
%\usepackage[pdftex]{colortbl}

% A nice monospace font for listings, etc.
\usepackage[scaled]{beramono}
%\usepackage{inconsolata}

% Using TikZ for diagrams
\usepackage{tikz}
\usetikzlibrary{shapes,mindmap}
%\usetikzlibrary{er,mindmap,calc,intersections,shapes,arrows,fit,matrix,positioning,decorations.pathmorphing,topaths,trees,automata}
\usepackage{tikz-cd} % for CM-arrow tips.

% Don't use externalize with gradients!!!
%\usetikzlibrary{external,arrows,fit,matrix,positioning}
%\tikzexternalize % Activate externalizing TikZ graphics.

\usepackage{pifont} % for ding
\usepackage[inline]{enumitem}
\usepackage{xspace}
\usepackage{listings}

% Scala listings.  Use colored Scala style by default.
\usepackage{lstscala}
\lstnewenvironment{lstscalasmall}{%
  \lstset{style=scala-color,basicstyle=\scriptsize\tt}}{}

\colorlet{ImportantCode}{ForestGreen}
\colorlet{ImportantCode2}{RubineRed}

\lstset{style=scala-color}

\lstnewenvironment{C}
  {\lstset{
    language=C,
    flexiblecolumns=false,
    mathescape=false,
    moredelim=**[is][\color{ImportantCode}]{@}{@},
    moredelim=**[is][\color{ImportantCode2}]{§}{§},
    basicstyle=\small\color{blue!30!darkgray}\tt,
commentstyle=\color{CadetBlue}}}
  {}
\lstnewenvironment{Scala}
  {\lstset{
    style=scala-color,
    flexiblecolumns=false,
    mathescape=false,
    basicstyle=\small\color{blue!30!darkgray}\tt}}
  {}


%%%% Custom macros %%%%
\newif\ifcompileTreeSlides
%\compileTreeSlidesfalse
\compileTreeSlidestrue

\newcommand{\SmallArrow}{\ding{228}}
\newcommand{\BigArrow}{$\longrightarrow$\xspace}
\newcommand{\Triangle}{$\triangleright$\xspace}
\renewcommand{\emph}[1]{\alert{#1}}
\newcommand{\light}{\color{TealBlue}}
\renewcommand{\hbar}{{\color{mLightBrown}\hrulefill}}

% http://tex.stackexchange.com/a/56585/77356
\tikzset{
  invisible/.style={opacity=0},
  visible on/.style={alt=#1{}{invisible}},
  alt/.code args={<#1>#2#3}{%
    \alt<#1>{\pgfkeysalso{#2}}{\pgfkeysalso{#3}} % \pgfkeysalso doesn't change the path
  },
}

%%% Document info %%%

\title{From Verified Functions to Safe C Code}

\author{Marco Antognini}

% To show the TOC at the beginning of each section, uncomment this:
% \AtBeginSection[]
% {
%   \begin{frame}<beamer>{Outline}
%     \tableofcontents[currentsection]
%   \end{frame}
% }

% To show the TOC at the beginning of each subsection, uncomment this:
% \AtBeginSubsection[]
% {
%   \begin{frame}<beamer>{Outline}
%     \tableofcontents[currentsection,currentsubsection]
%   \end{frame}
% }


% To uncover everything in a step-wise fashion, uncomment this:
% \beamerdefaultoverlayspecification{<+->}


\date{%
  \small Fall 2015\\[2em]
  \includegraphics[height=7mm]{img/epfl-logo}}


%%% Start of the actual document %%%

\begin{document}

\begin{frame}
  \titlepage
\end{frame}

% No outline, too short a talk...
% \begin{frame}{Outline}
%   \tableofcontents
%   % You might wish to add the option [pausesections]
% \end{frame}

\section{Motivation}

\begin{frame}[fragile]{Motivation}

  \begin{itemize}[label=\SmallArrow]

    \item Software keeps getting more complex.
    \item Hardware gets cheaper and more robust.
    \item For some industry, development cost is high mostly because of
      software verification process.
    \item Failure can have disastrous impact on human lives or huge monetary
      loss.

  \end{itemize}

  \vspace{1em}

  \BigArrow We need tools to \emph{verify complex software}.

\end{frame}

\begin{frame}[fragile]{Leon?}

  \begin{itemize}[label=\SmallArrow]

    \item Leon offers tools to reduce this cost.
    \item \emph{But!} It's JVM-based hence useless for \emph{very small
      devices} \\
      (e.g. pacemakers, spacecraft, \ldots)
    \item Need {\light native} code.
    \item That's why C-like languages are widely used in the industry.

  \end{itemize}

\end{frame}

\begin{frame}[fragile]{Verifier for C?}

  \begin{itemize}[label=\SmallArrow]
    \item Why not but low level, verbose and error prone.
    \item Modern C++ or D might solve many issues from C\ldots
    \item \ldots but no good verifiers exist.
  \end{itemize}

\end{frame}


\section{val GenC = (CAST) Leon}

\begin{frame}[fragile]{GenC}

  \begin{itemize}[label=\Triangle]
    \item Use Leon to {\light verify, repair and synthesis} code using high level
      features of Scala*. \pause
    \item Translate code into \emph{equivalent} C99 code. \pause
    \item Compile it with your favourite C compiler for your destination
      architecture. \pause
    \item And run it {\light natively} on your hardware.
  \end{itemize}

\end{frame}

\plain{\Huge Supported Features}

% Short-cut for inline listings using "@" (must come after "\titlepage")
\lstMakeShortInline[%
  style=scala-color,%
  flexiblecolumns=false,%
  mathescape=false,%
  basicstyle=\color{blue!30!darkgray}\tt]@

% Short-cut for inline listings using "|" (must come after "\titlepage")
\lstMakeShortInline[%
  language=C,%
  flexiblecolumns=false,%
  mathescape=false,%
  basicstyle=\color{blue!30!darkgray}\tt]|

\begin{frame}[fragile]{Types}

  @Int@ \BigArrow |int32_t| \hspace{3em} @Boolean@ \BigArrow |bool| \hspace{3em}
  @Unit@ \BigArrow |void|

  \pause
  \vspace{1em}

  @TupleN[T1, T2, ..., TN]@ is generated using some kind of templates. E.g.
  @(Boolean, Int)@ is:

  \begin{C}
typedef struct __leon_tuple_bool_int32_t_t {
  bool    @_1@;  // padding
  int32_t @_2@;
} __leon_tuple_bool_int32_t_t;
  \end{C}

  \pause
  %\vspace{1em}

  Similarly, @Array[T]@ is templatised. E.g. @Array[(Boolean, Int)]@ is:
  \begin{C}
typedef struct __leon_array___leon_tuple_bool_int32_t_t_t {
  __leon_tuple_bool_int32_t_t§*§ @data@;   // not owning the memory
  int32_t                      @length@;
} __leon_array___leon_tuple_bool_int32_t_t_t;
  \end{C}
\end{frame}

\begin{frame}[fragile]{More on Arrays}
  Those arrays live on the \emph{stack} \BigArrow cannot return them.

  {\light VLA} (variable-length array) are used when needed.

  \hbar

  \begin{Scala}
def foo(size: Int, value: Int) { val a = Array.fill(size)(value) }
  \end{Scala}

  \pause

  \vspace{-2em}
  \center{is translated into}
  \vspace{-0.5em}

  \begin{C}
void foo0(int32_t const size0, int32_t const value0) {
  int32_t __leon_vla_buffer0§[§size0§]§; // actual memory alloc
  for (int32_t __leon_i1 = 0; __leon_i1 < size0; ++__leon_i1) {
    __leon_vla_buffer0[__leon_i1] = value0;
  }
  __leon_array_int32_t_t const a3 =
    { @.length@ = size0, @.data@ = __leon_vla_buffer0 };
}
  \end{C}

\end{frame}

\begin{frame}[fragile]{Functions}
  \emph{Nested} functions are extracted with their context.

  \hbar

  \begin{Scala}
def foo(x: Int) = {
  def bar(y: Int) = x * y
  bar(2)
}
  \end{Scala}

  \pause

  \vspace{-2em}
  \center{is translated into}

  \begin{C}
int32_t bar0(int32_t const@*@ x0, int32_t const y6)
{ return @(*x0)@ * y6; }

int32_t foo0(int32_t const x0)
{ return bar0(@(&x0)@, 2); }
  \end{C}

\end{frame}

\begin{frame}[fragile]{Functions -- bis}

  \begin{Scala}
def foo(x: Int) = {
  def bar(y: Int) = {
    def fun(z: Int) = x * y + z
    fun(3)
  }
  bar(2)
}
  \end{Scala}

  \pause

  \vspace{-2em}
  \center{is translated into}
  \vspace{-0.5em}

  \begin{C}
int32_t fun0(int32_t const§*§ x0, int32_t const@*@ y6, int32_t const z9)
{ return @(*x0)@ * @(*y6)@ + z9; }

int32_t bar0(int32_t const@*@ x0, int32_t const y6)
{ return fun0(§x0§, @(&y6)@, 3); }

int32_t foo0(int32_t const x0)
{ return bar0(@(&x0)@, 2); }
  \end{C}

\end{frame}

\begin{frame}[fragile]{If construct}

  Unlike in Scala, |if| statements don't {\light return} a value. Hence

  \begin{columns}[T]
	  \column{.5\textwidth}
      \begin{Scala}
def foo(x: Int) {
  val b =
    if (x >= 0) true else false
}
      \end{Scala}

      \vspace{-2em}
      \center{is translated into}
      \vspace{-0.5em}

      \begin{C}
void foo0(int32_t const x0) {
  bool b0; // no const here
  if (x0 >= 0) { b0 = true; }
  else { b0 = false; }
}
      \end{C}

    \column{.5\textwidth}
      \pause

      \begin{onlyenv}<2->

      \begin{Scala}
def foo(x: Int) =
  if (x >= 0) true else false
      \end{Scala}

      \vspace{0.5em}
      \center{is translated into}
      \vspace{-0.5em}

      \begin{C}
bool foo1(int32_t const x0) {
  if (x0 >= 0) { return true; }
  else { return false; }
}
      \end{C}

      \end{onlyenv}

  \end{columns}

\end{frame}


\begin{frame}[fragile]{While construct}

  \begin{Scala}
def dummyAbs(a: Array[Int]) {
  var i = 0;
  while (i < a.length) {
    a(i) = if (a(i) < 0) -a(i) else a(i)
    i = i + 1
  }
}
  \end{Scala}

  \pause

  \vspace{-2.5em}
  \center{is translated into}
  \vspace{-0.5em}

  \begin{C}
void dummyAbs0(__leon_array_int32_t_t const a0) {
  int32_t i9 = 0;
  while (i9 < a0.length) {
    if (a0.data[i9] < 0) { a0.data[i9] = -a0.data[i9]; }
    else { a0.data[i9] = a0.data[i9]; }
    i9 = i9 + 1;
  }
}
  \end{C}

\end{frame}

\begin{frame}[fragile]{Subexpressions execution order}

  \begin{itemize}[label=\Triangle]
    \item C standard is much more lax when it comes to execution order.
    \item Hence the translation has to do some kind of \emph{normalisation}
      to ensure the same behaviour.
    \item This applies to function calls, operators, and blocks.
  \end{itemize}

\end{frame}

\begin{frame}[fragile]{Subexpressions execution order -- bis}

  \begin{columns}[T]
	  \begin{column}{.4\textwidth}

      \begin{Scala}
def test4(b: Boolean) = {
  var i = 10;
  var c = 0;
  val f = b && !b // false
  val t = b || !b // true

  while ({ c = c + 1; t } &&
         i > 0 ||
         { c = c * 2; f }) {
    i = i - 1
  }

  i == 0 && c == 22
}
      \end{Scala}

    \end{column}
    \pause

    {\color{mLightBrown}\vrule{}}

	  \begin{column}{.6\textwidth}

      \begin{C}
bool test40(bool const b0) {
  int32_t i10 = 10;
  int32_t c2 = 0;
  bool const f26 = b0 && (!b0);
  bool const t8  = b0 || (!b0);

  while (§true§) {
    c2 = c2 + 1;
    if (t8 && i10 > 0) { i10 = i10 - 1; }
    else {
      c2 = c2 * 2;
      if (f26) { i10 = i10 - 1; }
      else { break; }
    }
  }

  return i10 == 0 && c2 == 22;
}
      \end{C}

    \end{column}
  \end{columns}
\end{frame}

\plain{
  {\Huge Recap'}

  \begin{itemize}[label=\Triangle]
    \item Basic types
    \item Tuples \& stack allocated arrays
    \item Nested functions
    \item If \& While constructs
    \item Subexpressions orders
  \end{itemize}
}

\section{What's next}

\begin{frame}[fragile]{Future's work}
  \begin{itemize}[label=\SmallArrow]
    \item Support for non-recursive data type (@case class@).
    \item A case study about image processing.
  \end{itemize}

  \vspace{3em}
  \pause

  \begin{itemize}[label=\Triangle]
    \item Support for floating point types (|float|, |double|).
    \item Support for @BigInt@, @Real@, e.g. using GMP.
    \item Support for heap allocated arrays, e.g. using |malloc|.
  \end{itemize}
\end{frame}

\plain{\Huge Questions?}

\end{document}

% vim: spell spelllang=en_gb
% vim: set tabstop=2 softtabstop=2 shiftwidth=2 textwidth=80: %

