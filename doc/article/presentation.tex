\documentclass[10pt, compress, xcolor={usenames,dvipsnames}]{beamer}
\usepackage[utf8x]{inputenc}

%%% Theme and style
\usetheme{metropolis}
\metroset{progressbar=frametitle}
\metroset{numbering=fraction}

%%% Packages %%%

% Use T1 and a modern font family for better support of accents, etc.
\usepackage[T1]{fontenc}
\usepackage{palatino}  % Palatino

% Language support
\usepackage[english]{babel}

% Support for easily changing the enumerator in
% enumerate-environments.
\usepackage{enumerate}

% Support for importing images
%\usepackage{graphicx}

% Use hyperlinks
\usepackage{hyperref}

% Don't load xcolors package in beamer: use document class option
% instead...
%\usepackage[usenames,dvipsnames]{xcolor}

% Use colors in tables
%\usepackage[pdftex]{colortbl}

% A nice monospace font for listings, etc.
\usepackage[scaled]{beramono}
%\usepackage{inconsolata}

% Using TikZ for diagrams
\usepackage{tikz}
\usetikzlibrary{shapes,mindmap}
%\usetikzlibrary{er,mindmap,calc,intersections,shapes,arrows,fit,matrix,positioning,decorations.pathmorphing,topaths,trees,automata}
\usepackage{tikz-cd} % for CM-arrow tips.

% Don't use externalize with gradients!!!
%\usetikzlibrary{external,arrows,fit,matrix,positioning}
%\tikzexternalize % Activate externalizing TikZ graphics.

\usepackage{pifont} % for ding
\usepackage[inline]{enumitem}
\usepackage{xspace}
\usepackage{listings}

% Scala listings.  Use colored Scala style by default.
\usepackage{lstscala}
\lstnewenvironment{lstscalasmall}{%
  \lstset{style=scala-color,basicstyle=\scriptsize\tt}}{}

\colorlet{ImportantCode}{ForestGreen}
\colorlet{ImportantCode2}{RubineRed}

\lstset{style=scala-color}

\lstnewenvironment{C}
  {\lstset{
    language=C,
    flexiblecolumns=false,
    mathescape=false,
    moredelim=**[is][\color{ImportantCode}]{@}{@},
    moredelim=**[is][\color{ImportantCode2}]{§}{§},
    basicstyle=\small\color{blue!30!darkgray}\tt,
commentstyle=\color{CadetBlue}}}
  {}
\lstnewenvironment{Scala}
  {\lstset{
    style=scala-color,
    flexiblecolumns=false,
    mathescape=false,
    basicstyle=\small\color{blue!30!darkgray}\tt}}
  {}


%%%% Custom macros %%%%
\newif\ifcompileTreeSlides
%\compileTreeSlidesfalse
\compileTreeSlidestrue

\newcommand{\SmallArrow}{\ding{228}}
\newcommand{\BigArrow}{$\longrightarrow$\xspace}
\newcommand{\Triangle}{$\triangleright$\xspace}
\renewcommand{\emph}[1]{\alert{#1}}
\newcommand{\light}{\color{TealBlue}}
\renewcommand{\hbar}{{\color{mLightBrown}\hrulefill}}

% http://tex.stackexchange.com/a/56585/77356
\tikzset{
  invisible/.style={opacity=0},
  visible on/.style={alt=#1{}{invisible}},
  alt/.code args={<#1>#2#3}{%
    \alt<#1>{\pgfkeysalso{#2}}{\pgfkeysalso{#3}} % \pgfkeysalso doesn't change the path
  },
}

%%% Document info %%%

\title{An implementation of SGP4 in non-singular variables \\
              using a functional paradigm}

\author{Pablo Pita Leira}

% To show the TOC at the beginning of each section, uncomment this:
% \AtBeginSection[]
% {
%   \begin{frame}<beamer>{Outline}
%     \tableofcontents[currentsection]
%   \end{frame}
% }

% To show the TOC at the beginning of each subsection, uncomment this:
% \AtBeginSubsection[]
% {
%   \begin{frame}<beamer>{Outline}
%     \tableofcontents[currentsection,currentsubsection]
%   \end{frame}
% }


% To uncover everything in a step-wise fashion, uncomment this:
% \beamerdefaultoverlayspecification{<+->}


\date{%
  \small March 2016\\[2em]
%  \includegraphics[height=7mm]{img/epfl-logo}}


%%% Start of the actual document %%%

\begin{document}

\begin{frame}
  \titlepage
\end{frame}

% No outline, too short a talk...
% \begin{frame}{Outline}
%   \tableofcontents
%   % You might wish to add the option [pausesections]
% \end{frame}

\section{Motivation}

\begin{frame}[fragile]{SGP4}
Simplified General Perturbations 4) orbit propagator 

Inputs: two line elements diseminated by NORAD

widely used tool for the fast, short term propagation of earth satellite orbits

thoroughly described in the SPACETRACK report \#3 

numerous versions of SGP4: FORTRAN, C++, Java, MATLAB 

  \vspace{1em}


\end{frame}

\begin{frame}[fragile]{Why one more version}

SGP4Extensions

  \begin{itemize}[label=\SmallArrow]
  \item  It is heavily influenced by the functional software paradigm.
  \item  Equations have been expressed almost always literally writing the algebraic equations in the code as expressed in the papers
  \item  Implementations using other variables and/or extra terms can be easily introduced into the propagation algorithm
  \item  Provides more options and flexibility when being used within other algorithms, like those performing space debris conjunction analysis

  \end{itemize}

\end{frame}

\begin{frame}[fragile]{Scala}

  \begin{itemize}[label=\SmallArrow]
    \item Unicode support to express equations
	\item Lyddane 2nd Order Long Period Corrections:
  \end{itemize}

  \begin{Scala}
val δI =  ϵ3*esinω* c
val δa =  0
val δh = -ϵ3*ecosω* c/s
val δC = -ϵ3*ecosω* esinω * (1/s - 2*s)
val δS =  ϵ3*s - ϵ3*`η²`*s - 2*ϵ3*`ecosω²`*s + ϵ3*`ecosω²`/s
val δF =  ϵ3*s* ecosω * (1 - 2*`η²`/(1+η)) / 2
  \end{Scala}

\end{frame}

\begin{frame}[fragile]{Scala}


  \begin{itemize}[label=\SmallArrow]
    \item Hybrid object oriented/functional 
    \item Rich type system
    \item compiled to java byte code
    \item designed for creating DSL on top: expessive
  \end{itemize}

\end{frame}

\section{What's next}

\begin{frame}[fragile]{Future's work}
  \begin{itemize}[label=\SmallArrow]
    \item Propagation of the whole catalog
    \item Collision analysis
  \end{itemize}

  \vspace{3em}
  \pause

  \begin{itemize}[label=\Triangle]
    \item Support for floating point types (|float|, |double|).
    \item Support for @BigInt@, @Real@, e.g. using GMP.
    \item Support for heap allocated arrays, e.g. using |malloc|.
  \end{itemize}
\end{frame}

\plain{\Huge Questions?}

\end{document}

% vim: spell spelllang=en_gb
% vim: set tabstop=2 softtabstop=2 shiftwidth=2 textwidth=80: %

